\documentclass{tufte-handout}
\usepackage{graphicx}
\usepackage{exercise}
\usepackage{enumerate}
\usepackage{amsmath,amssymb,amsthm}
\usepackage{enumitem}
\newcommand{\bi}{\begin{itemize}}
\newcommand{\ei}{\end{itemize}}
\newcommand{\be}{\begin{enumerate}}
\newcommand{\ee}{\end{enumerate}}
\newcommand{\beq}{\begin{equation}}
\newcommand{\eeq}{\end{equation}}

\title{Module 1 Exercises: Interactions 2.2}
\begin{document}

\maketitle



\section{Statics}

Now that you've dealt with forces and moments,  you've done some drawing of free-body diagrams, and you've thought about distributed forces, let's apply these ideas to static conditions.

\begin{enumerate}[resume]


\begin{marginfigure} 
\includegraphics[width=6cm]{figs/Inter1_boxhang.jpg}
\end{marginfigure}

\item A box weighing 200 lbs is being supported by two cables, ab and cb. The angles of the cables with respect to the horizontal are shown. 
\be
\item Assume that the mass is stationary.  What is the tension in the cables?
\item Assume that the length of the cables is constant and the suspension points a and c are moved to the left and right, respectively. Do you expect the tension in the cables to increase, decrease, or stay the same? Do a calculation to verify your answer. 
\item What are three types of sensors can that be used to measure force? Identify a specific sensor that you could purchase that could be used to measure the tension predicted in cable ab in the previous question. Download a copy of that sensor data sheet or specifications and put a copy in your portfolio. 
\ee

\vfill
\begin{marginfigure} 
\includegraphics[width=6cm]{figs/BallOnTable}
\end{marginfigure}
\item Consider the system shown at the right.  It consists of a mass $m_1$ on a table, attached via a string to a mass $m_2$ beneath the table.  The system is initially at rest.  Assume that you can use a standard frictional model with a coefficient of static friction $\mu_s$ and a coefficient of kinetic friction $\mu_k$.



By drawing free body diagrams and applying statics conditions, determine the conditions (i.e., values of parameters $m_1$, $m_2$, $\mu_s$, and $\mu_k$) for which the system remains at rest.

\vfill
\begin{marginfigure} 
\includegraphics[width=6cm]{figs/MotorWeightStatics}
\end{marginfigure}

\item Consider the situation shown at the right.  There is a 1 kg weight sitting on a surface; the surface has a coefficients of static and kinetic friction $\mu_s$ and $\mu_k$.  The pulley is frictionless.  What is the minimum torque required for the motor to be able to move the weight (either by tipping it or dragging it)?  If you wanted to minimize the required torque, at what height would you put the 10 cm pulley?



\clearpage

\item Consider a block on a ramp, as shown. The ramp is free to move.  There is a frictional interaction between both the the block and the ramp and between the ramp and the ground.  \begin{marginfigure} 
\includegraphics[width=3in]{figs/BlockOnRampSimple}
\end{marginfigure}

\be
\item Consider the case that both the block and the ramp are stationary.  Find the necessary coefficients of friction to maintain this static condition.

\item Where is the equivalent contact force from the ground on the ramp applied?  How large and in what direction is this force?

\item Now consider the case that the block has started to slide down the ramp.  What are the necessary conditions for the ramp to remain stationary?  Where is the equivalent contact force from the ground on the ramp applied?  How large and in what direction is this force?

\ee

\vfill

 \begin{marginfigure} 
\includegraphics[width=2.5in]{figs/cablesprings.png}
\end{marginfigure}
\item Determine the weight of each block if they cause a (vertical) sag of d = 2 m when suspended the cable-springs. If the masses are removed, d = 0 (configuration in red). 

\vfill


 

 \begin{marginfigure} 
\includegraphics[width=2.5in]{figs/airplanewing.png}
\end{marginfigure}
\item Recall the the airplane wing from Interactions 2.1. The lift force varies as a function of position along the wing according to 
$$L(x) = 200 \sqrt{ 1-\frac{x^2}{17}}$$
where $L(x)$ is the force per unit length along the wing. The weight of the wing is 1600 N. Gravity points in the downward vertical direction.  Assume the wing is is fixed (no translation, no rotation) to the fuselage at point O (x=0), and that the center of mass of the wing is 2 meters from the fuselage. If the strut breaks, what are the reaction forces and moments that the airplane fuselage is applying at the root of the wing. What if the strut is not broken (this is a harder problem -- how should you deal with it)?  Consider the wing to be in static equilibrium. 


\clearpage
 \begin{marginfigure} 
\includegraphics[width=2.5in]{figs/floatingblock.pdf}
\end{marginfigure}

\item The diagram shows a block of blue foam that is $10 \times 10 \times d$ cm.  A lead weight of mass $m$ is attached to the center of one edge of the foam via a string; when the bock is placed in the water, it floats as shown (i.e., with two opposite edges at water level).   Assume the density of the foam is 20 $kg/m^3$, and the density of lead is 9.78 $g/cm^3$.  Determine a combination of $d$ and $m$ that lead to the foam floating in this way.

\vfill

\begin{marginfigure} 
\includegraphics[width=2in]{figs/beamonroller.png}
\end{marginfigure}

\item Consider the beam shown to the right.  What are the reactions at the supports of the beam? The support on the left is a pin connection. The support on the right is a roller (see the supports table). 
 
 \vfill
 \clearpage
 
 \begin{marginfigure} 
\includegraphics[width=2.5in]{figs/threeblocks.png}
\end{marginfigure}
\item (Warning: This problem is hard.  If you can't get it worked out, life will go on.) Three blocks are stacked on a surface that is inclined by an angle $\theta$. The mass of the blocks and coefficients of static friction between surfaces are shown. Gravity acts in the downward vertical direction. Plot the maximum (positive) value of F (if no slipping occurs) vs. $\theta$. What is the range of $F$ values over which the 50 kg block starts sliding alone and over which the 50 kg and 40 kg blocks start sliding together. 
 
 \vfill
 

\item Describe two real bodies or systems that are statically indeterminate.  Show why they are statically indeterminate. 

\vfill

\ee
\end{document}